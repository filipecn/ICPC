\documentclass{report}

\usepackage{geometry}
\geometry{a4paper,text={17cm,22.9cm,}}
\usepackage[brazil]{babel}
\usepackage[utf8]{inputenc}
\usepackage{multicol}
\usepackage{amsmath}
\usepackage{amssymb}
%\usepackage{mathptmx}
\usepackage{graphicx}
\usepackage{hyperref}
\usepackage{listings}
\usepackage{color}
\usepackage{here}

\newcommand \us {\underline \ }
\newcommand \tb {\hspace{-1.5em}}
\newcommand \degree {\hspace{-0.1em}\,^{\circ}}


\usepackage{listings}
\usepackage{courier}
\lstset{
	basicstyle=\small\ttfamily,
	numberstyle=\tiny,          
         stepnumber=1,               
         numbersep=5pt,            
         numbers=left,
         numberstyle=\small\ttfamily,
         tabsize=4,                  
         extendedchars=true, 
		inputencoding=latin1,
         breaklines=true, 
         keywordstyle=\color{red},
                frame=b,         
 %        keywordstyle=[1]\textbf,    %
 %        keywordstyle=[2]\textbf,    %
 %        keywordstyle=[3]\textbf,    %
 %        keywordstyle=[4]\textbf,   \sqrt{\sqrt{}} %
         stringstyle=\color{white}\ttfamily,
         showspaces=false,
         showtabs=false,             
         xleftmargin=17pt,
         framexleftmargin=17pt,
         framexrightmargin=5pt,
         framexbottommargin=4pt,
         showstringspaces=true         
 }

 \lstloadlanguages{
         %[Visual]Basic
         %Pascal
        C
         %C++
         %XML
         %HTML
         %Java
 }

\usepackage{caption}
\DeclareCaptionFont{white}{\color{white}}
\DeclareCaptionFormat{listing}{\colorbox[cmyk]{0.43, 0.35, 0.35,0.01}{\parbox{\textwidth}{\hspace{15pt}#1#2#3}}}
\captionsetup[lstlisting]{format=listing,labelfont=white,textfont=white, singlelinecheck=false, margin=0pt, font={bf,footnotesize}}

% This concludes the preamble

\begin{document}
\begin{titlepage}
\begin{center}
\Huge{\textsf{Universidade de São Paulo}}\\
\Large{\textsf{Instituto de Ciências Matemáticas e Computação}}
\end{center}
\end{titlepage}

\clearpage
\tableofcontents
\clearpage


\chapter{Geometria}
\section{Intersecções}
\subsection{Line $\times$ Line}
Apenas geomtria 2D, coisas q eu vou esquecer provavelmente ou ter q recalcular..\\
\begin{itemize}
\item{lei dos cosenos: $a^2 = b^2 + c^2 -2*b*c*cos(a)$}
\item{lei dos senos: $\frac{sin(a)}{a} = \frac{sin(b)}{b} = \frac{sin(c)}{c}$}
\item{cross: $|a*b| = |a|.|b|.sin(t)$}
\item{dot: $a.b = |a|.|b|.cos(t)$}
\end{itemize}

%\lstinputlisting[label=lineinter,caption=linhas na forma ax + by + c = 0]{codes/geometry/lineinter.c}
%\lstinputlisting[label=lineinter2,caption=na forma = s*u+v; u = (p2-p1)/norm; v = p1]{codes/geometry/lineinter2.c}

\subsection{Circle $\times$ Circle}
%\lstinputlisting[label=circlecircle,caption=Intersecao de circulos, usando os triangulos que formam ]{codes/geometry/circlelinecirlce.c}
\subsection{Circle $\times$ Circle : Area}

%\lstinputlisting[label=circlearea,caption=area intersecao de circulos, usando os triangulos que formam entre eles]{codes/geometry/circlearea.c}

\subsection{Circle $\times$ Line}
 Dado dois pontos da linha calcula a intersecao com circulo centrado em 0,0
   caso precise deslocar o centro do circulo pra 0,0.. 
%\lstinputlisting[label=circleline,caption=asd]{codes/geometry/circleline.c}
\subsection{Circle Triple}
 Dado dois circulos e o raio de um terceiro, acha a posicao que colide com os outros 2, 
   ideia de desenhar os circulos com raio r1+r e r2+r e achar a intersecao e la fica o circ! 
%\lstinputlisting[label=circletriple,caption=asd]{codes/geometry/circletriple.c}
\section{Distância Ponto Segmento}
%\lstinputlisting[label=pointsegment,caption=asd]{codes/geometry/pointsegment.c}
\section{Convex Hull}
\chapter{Grafos}

\chapter{Teoria dos Números}

\end{document}
